\documentclass[final]{beamer}

\usepackage[scale=1.24]{beamerposter}
\usetheme{confposter}
\setbeamercolor{block title}{fg=ngreen,bg=white}
\setbeamercolor{block body}{fg=black,bg=white}
\setbeamercolor{block alerted title}{fg=white,bg=dblue!70}
\setbeamercolor{block alerted body}{fg=black,bg=dblue!10}

\newlength{\sepwid}
\newlength{\onecolwid}
\newlength{\twocolwid}
\newlength{\threecolwid}
\setlength{\paperwidth}{48in} % A0 width: 46.8in
\setlength{\paperheight}{34in} % A0 height: 33.1in
\setlength{\sepwid}{0.024\paperwidth} % Separation width (white space) between columns
\setlength{\onecolwid}{0.22\paperwidth} % Width of one column
\setlength{\twocolwid}{0.464\paperwidth} % Width of two columns
\setlength{\threecolwid}{0.708\paperwidth} % Width of three columns
\setlength{\topmargin}{-0.5in} % Reduce the top margin size
%-----------------------------------------------------------

\usepackage{graphicx}  % Required for including images

\usepackage{booktabs} % Top and bottom rules for tables

\usepackage{hyperref}
\hypersetup{
    colorlinks=true,
    linkcolor=blue,
    filecolor=magenta,      
    urlcolor=blue,
}

\usepackage{amsmath}

\title{Close Encounter: 2017 MZ8 } % Poster title

\author{Bosscha Observatory $\vert$ Astronomy Research Division, ITB}

\institute{This information is generated on 2017-09-07 16:47 UTC.} 

\begin{document}

\addtobeamertemplate{block end}{}{\vspace*{2ex}}
\addtobeamertemplate{block alerted end}{}{\vspace*{2ex}}

\setlength{\belowcaptionskip}{2ex}
\setlength\belowdisplayshortskip{2ex}

\begin{frame}[t]

\begin{columns}[t] 

% First Column

\begin{column}{\sepwid}\end{column}

\begin{column}{\onecolwid}

\begin{alertblock}{Basic Properties}
\begin{itemize}
\item Name: 2017 MZ8 
\item Estimated diameter: 99. \--- 222 meters
\item Classification: Apollo (NEO)
\item H: 22.136
\item Period: 1440.09 days
\end{itemize}
\begin{table}
\caption{Orbital elements at epoch 2457927.5 JD }
\begin{tabular}{l c r}
\toprule
\textbf{Parameter} & & \textbf{    Value    } \\
\midrule Semi-major axis ($a$) & & 2.49572 \\ 
Eccentricity ($e$) & & .649219 \\ 
Inclination ($i$) & & 4.61183 \\ 
Lon. of ascending node ($\Omega$) & & 17.0602 \\ 
Argument of pericenter ($\omega$) & & 345.426 \\ 
Mean Anomaly ($M$) & & 343.002 \\ 
\bottomrule
\end{tabular}
\end{table}

\end{alertblock}


\begin{block}{Orbit}
\begin{figure}
\includegraphics[width=0.99\textwidth]{initial_orbit.pdf}
\caption{Orbit of the asteroid and the planets. This plot only shows the inner region of the Solar System.}
\end{figure}
\end{block}

\begin{alertblock}{Acknowledgements}
Our script retrieve basic informations and initial condition from \href{https://ssd.jpl.nasa.gov/}{JPL NASA}, re-integrate the Solar System using \href{https://github.com/hannorein/rebound}{\texttt{Rebound}} package, and write this report using \LaTeX.
\end{alertblock}

\end{column}

% Second column

\begin{column}{\sepwid}\end{column}

\begin{column}{\twocolwid}

\begin{block}{Distance to the Earth and Orbital Elements}
\begin{figure}
\includegraphics[width=0.99\linewidth]{daeiw.pdf}
\caption{Distance to the Earth and orbital elements of the asteroid for the next 100 years. Starting time of the integration is 2017-08-17 00:00 UTC.}
\end{figure}
\end{block}

\begin{columns}[t, totalwidth=\twocolwid] % Split up the two columns wide column

\begin{column}{\onecolwid}\vspace{-.6in}
\begin{block}{Geocentric}
\begin{figure}
\includegraphics[width=0.99\textwidth]{geocentric.pdf}
\caption{Movement of the asteroid relative to the Earth ($xy$-plane; only the first 12 years of integration).}
\end{figure}
\end{block}
\end{column}

\begin{column}{\onecolwid}\vspace{-.6in} 
\begin{block}{Rotating Frame}
\begin{figure}
\includegraphics[width=0.99\textwidth]{rotframe.pdf}
\caption{Movement of the asteroid relative to the Sun-Earth system ($xy$-plane; only the first 12 years of integration).}
\end{figure}
\end{block}
\end{column}

\end{columns} 

\end{column} 


%%%% 
\begin{column}{\sepwid}\end{column} 

\begin{column}{\onecolwid} 

\setbeamercolor{block alerted title}{fg=black,bg=norange}
\setbeamercolor{block alerted body}{fg=black,bg=norange!10}

\begin{alertblock}{Nearest close encounter}
\begin{itemize}
\item Time: 1 September 2017, 12.05 UT
\item Distance:  $ 7.066 \times 10^6$ km ($18.38$ Lunar Distance)
\item Relative velocity: 
\end{itemize}
\end{alertblock}


\begin{alertblock}{List of close encounter}
\begin{table}
\vspace{2ex}
\begin{tabular}{l c c c}
\toprule
\textbf{Time} & \textbf{   Body   } & \textbf{  $\boldsymbol{d}$ (au)  } & \textbf{$\boldsymbol{v_{rel}}$ (km/s)} \\
\midrule\bottomrule
\end{tabular}
%\caption{Word Formation}
\end{table}
\end{alertblock}

\end{column} 


\end{columns} 
\end{frame}
\end{document}